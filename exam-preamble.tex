\documentclass{report}

% need to use this to get the right text according to the old style guides for UiT
% https://github.com/sebschub/FontPro/
% \usepackage{MyriadPro}

% to get helvetica fonts (similar to the myriad) in pdflatex. 
% http://tex.stackexchange.com/questions/121061/working-with-arial-or-helvetica-fonts
\usepackage[scaled]{helvet}
\renewcommand\familydefault{\sfdefault} 
\usepackage[T1]{fontenc}


%\usepackage[scaled=1.04]{couriers} % to enable boldface code
\usepackage[scaled=0.9]{couriers} % to enable boldface code

\usepackage{graphicx}

\usepackage[lastpage,user]{zref}
\usepackage{color}
\usepackage[utf8]{inputenc}  % could use xelatex instead, but doesn't seem to work? 
\usepackage{fancyhdr}
\usepackage{enumerate}


\usepackage{tikz}
\usetikzlibrary{calc}   % for absolute positioning of figures on page

%\usepackage{tikz-uml}  % for uml drawings 

% The word file has margins: top 1.36cm, left 2.85cm, bottom 2.5cm right 2.5cm 
% http://www.sharelatex.com/learn/Page_size_and_margins
\usepackage[a4paper, margin=1cm]{geometry}
\geometry{
  a4paper, 
  total = {210mm, 297mm}, 
  left = 25mm, 
  right = 25mm, 
  top = 13mm, 
  bottom  = 25mm,
}

% distance between paragraphs 
\setlength{\parskip}{\baselineskip}%
\setlength{\parindent}{0pt}%


% settings for code snippets
\usepackage{listings}
\usepackage{fancyvrb}
%        commentstyle=\color{blue},
\lstset{language=Python,
        basicstyle=\footnotesize\ttfamily,
        frame=none,
        stringstyle=\color{blue},
        fancyvrb=true,
        xleftmargin=0pt,xrightmargin=2pt,
        showstringspaces=false}
% Fixes norwegian chars for lstlisting except for one thing:
% The Python code can't start with a norwegian character. 
% Ending with a norwegian character is ok. 
%\lstset{literate = %
%  {æ}{\ae}1
%  {ø}{\o}1
%  {å}{\aa}1
%  {Æ}{\AE}1
%  {Ø}{\O}1
%  {Å}{\AA}1
%}
% Better one from 
% http://en.wikibooks.org/wiki/LaTeX/Source_Code_Listings
\lstset{literate=
  {á}{{\'a}}1 {é}{{\'e}}1 {í}{{\'i}}1 {ó}{{\'o}}1 {ú}{{\'u}}1
  {Á}{{\'A}}1 {É}{{\'E}}1 {Í}{{\'I}}1 {Ó}{{\'O}}1 {Ú}{{\'U}}1
  {à}{{\`a}}1 {è}{{\`e}}1 {ì}{{\`i}}1 {ò}{{\`o}}1 {ù}{{\`u}}1
  {À}{{\`A}}1 {È}{{\'E}}1 {Ì}{{\`I}}1 {Ò}{{\`O}}1 {Ù}{{\`U}}1
  {ä}{{\"a}}1 {ë}{{\"e}}1 {ï}{{\"i}}1 {ö}{{\"o}}1 {ü}{{\"u}}1
  {Ä}{{\"A}}1 {Ë}{{\"E}}1 {Ï}{{\"I}}1 {Ö}{{\"O}}1 {Ü}{{\"U}}1
  {â}{{\^a}}1 {ê}{{\^e}}1 {î}{{\^i}}1 {ô}{{\^o}}1 {û}{{\^u}}1
  {Â}{{\^A}}1 {Ê}{{\^E}}1 {Î}{{\^I}}1 {Ô}{{\^O}}1 {Û}{{\^U}}1
  {œ}{{\oe}}1 {Œ}{{\OE}}1 {æ}{{\ae}}1 {Æ}{{\AE}}1 {ß}{{\ss}}1
  {ç}{{\c c}}1 {Ç}{{\c C}}1 {ø}{{\o}}1 {å}{{\r a}}1 {Å}{{\r A}}1
  {€}{{\EUR}}1 {£}{{\pounds}}1
  % JMB addons
  {Ø}{{\O}}1
}

\lstset{ % 
  basicstyle=\ttfamily, % Otherwise, bfseries won't work
  %keywordstyle=\bfseries \color{blue}, 
  keywordstyle=\bfseries, 
  commentstyle=\color{blue}
}

%morekeywords={include, printf}


%\newcommand{\py}[1]{\lstinline|#1|}
\newcommand{\py}[1]{\lstinline{#1}}

\usepackage{comment}  % for leaving out eguide sections

% ---------------------------------------------------------------------------------------
% Get parameters and set useful params

\def\examCourseName{INF-1400 Objektorientert Programmering}
\def\examDate{Tirsdag 26. mai 2015}
\def\examTime{Kl 09:00 - 13:00}
\def\examPlace{Åsgårdveien 9}
\def\examLecturer{John Markus Bjørndalen}
\def\examPhone{90148307}

% only use one of the following defines. 
\def\examLangNo{1}  % Norsk bokmål
%\def\examLangEng{1} % English



\ifdefined\examLangNo
\def\examDefExa{Eksamen i}
\def\examDefExb{For eksamen i}
\def\examDateName{Dato}
\def\examTimeName{Tid}
\def\examPlaceName{Sted}
\def\examPhoneName{Telefon}
\def\examApprovedHelp{Tillatte hjelpemidler: Ingen}
\def\examNoScratch{NB! Det er ikke tillatt å levere inn kladd sammen med besvarelsen}
\def\examLogoa{logo/logo-uitnavnetrekk-cropped.pdf}
\def\examLogob{logo/Logo_norsk_pos_crop.pdf}
\def\examNpagesA{Oppgavesettet er på}
\def\examNpagesB{sider inklusiv forside}
\def\examGuideNPA{Sensorveiledningen er på}
\def\examGuideNPB{sider inklusiv forside}
\def\examLecturerLead{Fagperson/intern sensor}
\def\examContactName{Kontaktperson under eksamen}
\def\examMainTitle{EKSAMENSOPPGAVE}
\def\examGuideTitle{SENSORVEILEDNING}
\fi     

\ifdefined\examLangEng
\def\examDefExa{Exam in}
\def\examDefExb{For exam in}
\def\examDateName{Date}
\def\examTimeName{Time}
\def\examPlaceName{Place}
\def\examPhoneName{Phone}
\def\examApprovedHelp{Approved aids: None}
\def\examNoScratch{NB! It is not allowed to submit scratch paper along with the answer sheets}
\def\examLogoa{logo/logo-uitnavnetrekk-eng-cropped.pdf}
\def\examLogob{logo/Logo_eng_pos_crop.pdf}
\def\examNpagesA{The exam contains}
\def\examNpagesB{pages, including this cover page} % added a comma as well. Clarifies the meaning (without the comma it sounds like we have X pages that include this cover page)
\def\examGuideNPA{The assessment guideline contains}
\def\examGuideNPB{pages, including this cover page}
\def\examLecturerLead{Contact person}
\def\examContactName{Contact person}
\def\examMainTitle{EXAMINATION PAPER}
\def\examGuideTitle{ASSESSMENT GUIDELINE}
\fi     

\ifdefined\examIsGuide
\def\examTitle{\examGuideTitle}
\else
\def\examTitle{\examMainTitle}
\fi


% ------------------------------------------------------------  
% Front page stuff

\begin{document}

% This is more flexible than creating the front page in word and adding it here. 
\begin{titlepage}

\begin{tikzpicture}[remember picture,overlay]
  \node[anchor=north west,inner sep=0pt] at ($(current page.north west)+(1cm,-1cm)$) {
     \includegraphics[width=3cm]{\examLogoa} 
  };
\end{tikzpicture}

\vspace{5.5cm}

\begin{center} 
{\Huge \textbf{\examTitle}}
\end{center}

\vspace{1cm}
\large


\ifdefined\examIsGuide
%------------------------------------------------------------ Guide --------
% @{} before the first column removes the small indentation that it gets automatically
\textbf{
  \begin{tabular}{@{}ll} \\
    \examDefExb: &  \examCourseName \\
    \examDateName:        &  \examDate \\
  \end{tabular}
}
\vspace{4cm}

\textbf{\examGuideNPA\ \zpageref{LastPage}\ \examGuideNPB}

\vspace{1cm}

\textbf{
\begin{tabular}{@{}l}  \\
\examLecturerLead: \examLecturer. \\
\examPhoneName: \examPhone   % her burde det ha vært e-mail også. 
\end{tabular}
}

\else        %------------------------------------------------------------ Exam --------
% @{} before the first column removes the small indentation that it gets automatically
\textbf{
\begin{tabular}{@{}ll} \\
\examDefExa:     & \examCourseName \\
\examDateName:   & \examDate \\
\examTimeName: 	 & \examTime \\ 
\examPlaceName:	 & \examPlace
\end{tabular}
}

\vspace{1cm}
\textbf{
\examApprovedHelp
}
\vspace{1cm}

% added a comma as well. Clarifies the meaning (without the comma it sounds like we have X pages that include this cover page)
\textbf{
\examNpagesA\ \zpageref{LastPage}\ \examNpagesB.	
}

\vspace{1cm}

\textbf{
\begin{tabular}{@{}l}  \\
\examContactName: \examLecturer. \\
\examPhoneName: \examPhone
\end{tabular}
}

\vspace{2cm}

\textbf{\textcolor{red}{\examNoScratch}}
%\textbf{\examNoScratch}
\fi % ------------------------------------------------------------   diff end

\begin{tikzpicture}[remember picture,overlay]
  \node[anchor=south east,inner sep=0pt] at ($(current page.south east)+(-2cm,2cm)$) {
     \includegraphics[width=2cm]{\examLogob} 
  };
\end{tikzpicture}

\end{titlepage}


%% Defining the eguide environment for exam guides (empty if exam)
\ifdefined\examIsGuide
  \newsavebox{\eguidesavebox}
  \newenvironment{eguide}{
    \begin{lrbox}{\eguidesavebox}
      %\begin{minipage}{0,99\textwidth}
      \begin{minipage}{0,99\linewidth} % suggestion from Weihai
  }
  {
      \end{minipage}
    \end{lrbox}
  \fcolorbox{blue!30}{gray!7}{\usebox{\eguidesavebox}}
  }
\else
  \excludecomment{eguide}
\fi


%% Settings before starting with the main exam text 
\setcounter{page}{2}  % title page should count as page 1

%\pagebreak  % no need after titlepage
\pagestyle{fancy}
\fancyhf{}
\renewcommand{\headrulewidth}{0pt} % removes line at the top of the page

\lfoot{\small  UiT / Postboks 6050 Langnes, N-9037 Tromsø / 77 64 40 00 /  postmottak@uit.no / uit.no}
\rfoot{\small \thepage / \zpageref{LastPage}}
\cfoot{}
