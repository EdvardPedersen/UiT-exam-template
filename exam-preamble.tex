\documentclass{report}

% need to use this to get the right text according to the old style guides for UiT
% https://github.com/sebschub/FontPro/
% \usepackage{MyriadPro}

% to get helvetica fonts (similar to the myriad) in pdflatex. 
% http://tex.stackexchange.com/questions/121061/working-with-arial-or-helvetica-fonts
%\usepackage[scaled]{helvet}
% 2019 - the styles now use times new roman
%\usepackage{times}
%\usepackage{mathptmx}
\usepackage{newtxtext}
%\renewcommand\familydefault{\sfdefault} 
\usepackage[T1]{fontenc}

% for \fontsize local font size
\usepackage{anyfontsize}

% for local line separation (setstretch)
\usepackage{setspace}

%\usepackage[scaled=1.04]{couriers} % to enable boldface code
\usepackage[scaled=0.9]{couriers} % to enable boldface code

\usepackage{graphicx}

\usepackage[lastpage,user]{zref}
\usepackage{color}
\usepackage[utf8]{inputenc}  % could use xelatex instead, but doesn't seem to work? 
\usepackage{fancyhdr}
\usepackage{enumerate}

% minipage environments have problems spanning pages,
% trying this as an alternative
% http://texdoc.net/texmf-dist/doc/latex/tcolorbox/tcolorbox.pdf
%\usepackage[skins,breakable]{tcolorbox}
\usepackage[most]{tcolorbox}

\usepackage{tikz}
\usetikzlibrary{calc}   % for absolute positioning of figures on page

%\usepackage{tikz-uml}  % for uml drawings 

% The word file has margins: top 1.36cm, left 2.85cm, bottom 2.5cm right 2.5cm 
% http://www.sharelatex.com/learn/Page_size_and_margins
\usepackage[a4paper, includefoot, margin=1cm]{geometry}
\geometry{
  a4paper, 
  total = {210mm, 297mm}, 
  left = 25mm, 
  right = 25mm, 
  top = 13mm, 
  bottom  = 25mm,
}

% distance between paragraphs 
\setlength{\parskip}{\baselineskip}%
\setlength{\parindent}{0pt}%


% settings for code snippets
\usepackage{listings}
\usepackage{fancyvrb}
%        commentstyle=\color{blue},
\lstset{language=Python,
        basicstyle=\footnotesize\ttfamily,
        frame=none,
        stringstyle=\color{blue},
        fancyvrb=true,
        xleftmargin=0pt,xrightmargin=2pt,
        showstringspaces=false}
% Fixes norwegian chars for lstlisting except for one thing:
% The Python code can't start with a norwegian character. 
% Ending with a norwegian character is ok. 
%\lstset{literate = %
%  {æ}{\ae}1
%  {ø}{\o}1
%  {å}{\aa}1
%  {Æ}{\AE}1
%  {Ø}{\O}1
%  {Å}{\AA}1
%}
% Better one from 
% http://en.wikibooks.org/wiki/LaTeX/Source_Code_Listings
\lstset{literate=
  {á}{{\'a}}1 {é}{{\'e}}1 {í}{{\'i}}1 {ó}{{\'o}}1 {ú}{{\'u}}1
  {Á}{{\'A}}1 {É}{{\'E}}1 {Í}{{\'I}}1 {Ó}{{\'O}}1 {Ú}{{\'U}}1
  {à}{{\`a}}1 {è}{{\`e}}1 {ì}{{\`i}}1 {ò}{{\`o}}1 {ù}{{\`u}}1
  {À}{{\`A}}1 {È}{{\'E}}1 {Ì}{{\`I}}1 {Ò}{{\`O}}1 {Ù}{{\`U}}1
  {ä}{{\"a}}1 {ë}{{\"e}}1 {ï}{{\"i}}1 {ö}{{\"o}}1 {ü}{{\"u}}1
  {Ä}{{\"A}}1 {Ë}{{\"E}}1 {Ï}{{\"I}}1 {Ö}{{\"O}}1 {Ü}{{\"U}}1
  {â}{{\^a}}1 {ê}{{\^e}}1 {î}{{\^i}}1 {ô}{{\^o}}1 {û}{{\^u}}1
  {Â}{{\^A}}1 {Ê}{{\^E}}1 {Î}{{\^I}}1 {Ô}{{\^O}}1 {Û}{{\^U}}1
  {œ}{{\oe}}1 {Œ}{{\OE}}1 {æ}{{\ae}}1 {Æ}{{\AE}}1 {ß}{{\ss}}1
  {ç}{{\c c}}1 {Ç}{{\c C}}1 {ø}{{\o}}1 {å}{{\r a}}1 {Å}{{\r A}}1
  {€}{{\EUR}}1 {£}{{\pounds}}1
  % JMB addons
  {Ø}{{\O}}1
}

\lstset{ % 
  basicstyle=\ttfamily, % Otherwise, bfseries won't work
  %keywordstyle=\bfseries \color{blue}, 
  keywordstyle=\bfseries, 
  commentstyle=\color{blue}
}

%morekeywords={include, printf}


%\newcommand{\py}[1]{\lstinline|#1|}
\newcommand{\py}[1]{\lstinline{#1}}

\usepackage{comment}  % for leaving out eguide sections

% ---------------------------------------------------------------------------------------
% Get parameters and set useful params

\def\examCourseName{INF-1400 Objektorientert Programmering}
\def\examDate{Tirsdag 26. mai 2015}
\def\examTime{Kl 09:00 - 13:00}
\def\examPlace{Åsgårdveien 9}
\def\examLecturer{John Markus Bjørndalen}
\def\examPhone{90148307}

% only use one of the following defines. 
\def\examLangNo{1}  % Norsk bokmål
%\def\examLangEng{1} % English


% NB: they use the word rough paper for scratch/draft paper. 

\ifdefined\examLangNo
\def\examDefExa{Eksamen i}
\def\examDefExb{Kurskode}
\def\examDateName{Dato}
\def\examDateNameb{Eksamensdato}
\def\examTimeName{Klokkeslett}
\def\examPlaceName{Sted}
\def\examPhoneName{Telefon/mobil}
\def\examPhoneNameb{Telefon/e-mail}
\def\examApprovedHelp{Tillatte \newline hjelpemidler}
% couldn't get doublespacing or other things from the setspace package to work, so we're using a manual fix for now (deadline approaching)
\def\examNoScratch{NB! Det er ikke tillatt å levere inn kladdepapir som del av eksamensbesvarelsen. \vspace{0.2cm} \newline Hvis det likevel leveres inn, vil kladdepapiret bli holdt tilbake og ikke bli sendt til sensur.}
\def\examLogoa{logo/logo-uitnavnetrekk-cropped.pdf}
\def\examLogob{logo/Logo_norsk_pos_crop.pdf}
\def\examNpagesA{Oppgavesettet er på}
\def\examNpagesB{sider inklusiv forside}
\def\examGuideNP{Antall sider \newline inkl. forside}
\def\examLecturerLead{Fagperson/intern sensor}
\def\examContactName{Kontaktperson under \newline eksamen}
\ifdefined\examVisitTime
\def\examVisit{Vil det bli gått oppklaringsrunde i eksamenslokalet? \newline \textbf{JA, ca. kl.: \examVisitTime \newline NEI: }}
\else
\def\examVisit{Vil det bli gått oppklaringsrunde i eksamenslokalet? \newline \textbf{JA, ca. kl.: \newline NEI: X }}
\fi
\def\examMainTitle{EKSAMENSOPPGAVE}
\def\examGuideTitle{SENSORVEILEDNING}
\def\examTPRhead{Fakultet for naturvitenskap og teknologi}
\def\examTypePaper{Type \newline innføringsark (rute/linje)}
\def\examNumPagesInfo{Antall sider \newline inkl. forside}
\fi     

\ifdefined\examLangEng
\def\examDefExa{Exam in}
\def\examDefExb{For exam in}
\def\examDateName{Date}
\def\examDateNameb{Date}
\def\examTimeName{Time}
\def\examPlaceName{Place}
\def\examPhoneName{Phone}
\def\examPhoneNameb{Phone/e-mail}
\def\examApprovedHelp{Approved aids}
\def\examNoScratch{NB! It is not allowed to submit scratch paper along with the answer sheets. \newline If you do submit scratch paper, it will not be evaluated.}
\def\examLogoa{logo/logo-uitnavnetrekk-eng-cropped.pdf}
\def\examLogob{logo/Logo_eng_pos_crop.pdf}
\def\examNpagesA{The exam contains}
\def\examNpagesB{pages, including this cover page} % added a comma as well. Clarifies the meaning (without the comma it sounds like we have X pages that include this cover page)
\def\examGuideNP{Contains pages: \newline including this cover page}
\def\examLecturerLead{Contact person}
\def\examContactName{Contact \newline person during \newline the exam}
\ifdefined\examVisitTime
\def\examVisit{Will a lecturer/person in charge visit the venue? \newline \textbf{YES, approx. time: \examVisitTime \newline NO:} }
\else
\def\examVisit{Will a lecturer/person in charge visit the venue? \newline \textbf{YES, approx. time:  \newline NO: X} }
\fi
\def\examMainTitle{EXAMINATION QUESTION PAPER} 
\def\examGuideTitle{ASSESSMENT GUIDELINE}
\def\examTPRhead{Faculty of Science and Technology} 
\def\examTypePaper{Type of sheets \newline (squares/lines)}
\def\examNumPagesInfo{Number of \newline pages incl. \newline cover page}
\fi     

\ifdefined\examIsGuide
\def\examTitle{\examGuideTitle}
\else
\def\examTitle{\examMainTitle}
\fi


% ------------------------------------------------------------  
% Front page stuff

\begin{document}

% This is more flexible than creating the front page in word and adding it here. 
\begin{titlepage}

\pagestyle{fancy}
\fancyhf{}
\renewcommand{\headrulewidth}{0pt} % removes line at the top of the page
\lfoot{\small  PO Box 6050 Langnes, NO-9037 Tromsø / +47 77 64 40 00 /  postmottak@uit.no / uit.no}
\rhead{\small \examTPRhead}
\headheight 40pt
\thispagestyle{fancy}

\begin{tikzpicture}[remember picture,overlay]
  \node[anchor=north west,inner sep=0pt] at ($(current page.north west)+(1cm,-1cm)$) {
     \includegraphics[width=3cm]{\examLogoa} 
  };
\end{tikzpicture}

\vspace{1.0cm}


\ifdefined\examIsGuide
%------------------------------------------------------------ Guide --------
%\begin{center}
\vspace{2cm}

{\Huge \textbf{\examTitle}}
%\end{center}

\vspace{0.4cm}
%\large

% @{} before the first column removes the small indentation that it gets automatically
%\textbf{
\begin{tabular}{@{}p{3cm}l} \\
  \examDefExb:    &  \examCourseName \\ [0.5cm]
  \examDateNameb:  &  \examDate \\ [0.5cm]
  \examLecturerLead: & \examLecturer. \\ [1.0cm]
  \examPhoneNameb: & \examPhone  \\  [0.5cm]
  \examGuideNP  &  \zpageref{LastPage} \\ [0.5cm]
\end{tabular}
\vspace{4cm}
%}

\vspace{1cm}


\else        %------------------------------------------------------------ Exam --------
\vspace{1.5cm}

\begin{center}
  % fontsize,baselineskip (set bs to about 1.2x fontsize)
{\fontsize{27}{33}\selectfont \textbf{ \examTitle}}
\end{center}

\vspace{0.1cm}
\large


% @{} before the first column removes the small indentation that it gets automatically
{\def\arraystretch{2}\tabcolsep=10pt
\begin{tabular}{@{}|p{2.5cm}|p{11.5cm}|} \hline
\examDefExa:           & \examCourseName \\ \hline
\examDateName:         & \examDate \\ \hline
\examTimeName: 	       & \examTime \\ \hline
\examPlaceName:	       & \examPlace \\ \hline
\examApprovedHelp:     & \examApprovedHelpList \\ \hline
\examTypePaper:        & \examTypePaperAllowed \\ \hline
\examNumPagesInfo:     & \zpageref{LastPage} \\ \hline
\examContactName:      & \examLecturer \\
% line between contact name and phone only for english template
\ifdefined\examLangEng
\hline
\fi
\examPhoneName:        & \examPhone \\ \hline
% do not span columns for english template
\ifdefined\examLangEng
                       & \examVisit \\ \hline
% but span two columns for norwegian template
\else
\multicolumn{2}{|p{15cm}|}{\examVisit} \\ \hline
\fi
\end{tabular}
}


\textbf{\textcolor{red}{\examNoScratch}}

%\textbf{\examNoScratch}
\fi % ------------------------------------------------------------   diff end (exam or guide)

\begin{tikzpicture}[remember picture,overlay]
  \node[anchor=south east,inner sep=0pt] at ($(current page.south east)+(-1cm,1.2cm)$) {
     \includegraphics[width=1.5cm]{\examLogob} 
  };
\end{tikzpicture}

%{\small  UiT / Postboks 6050 Langnes, N-9037 Tromsø / 77 64 40 00 /  postmottak@uit.no / uit.no}
\end{titlepage}


%% Defining the eguide environment for exam guides (empty if exam)
\ifdefined\examIsGuide
  \newsavebox{\eguidesavebox}
  \newenvironment{eguide}{
      \begin{tcolorbox}[width=0.99\linewidth,breakable]
  }
  {
      \end{tcolorbox}
  }
\else
  \excludecomment{eguide}
\fi


%% Settings before starting with the main exam text 
\setcounter{page}{2}  % title page should count as page 1

%\pagebreak  % no need after titlepage
\pagestyle{fancy}
\fancyhf{}
\renewcommand{\headrulewidth}{0pt} % removes line at the top of the page

\lfoot{\small  UiT / Postboks 6050 Langnes, N-9037 Tromsø / 77 64 40 00 /  postmottak@uit.no / uit.no}
\rfoot{\small \thepage / \zpageref{LastPage}}
\cfoot{}
